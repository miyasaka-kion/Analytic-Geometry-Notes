\documentclass[11pt]{book}
\title{Geometric Algebra Pen Note}
\author{Kion Conio}
\date{\today}


\begin{document}	
\maketitle
\chapter{Vector algebra}
\section{The Algebra of vectors}

\subsection{Vectors and its Algebraic Operations }

\subsubsection{Vectors}
\paragraph{Definition}
A quantity with both magnitude and direction is called a vector. e.g. Force, velocity, acceleration, displacement, etc.

\paragraph{Notation}
Directed line segment: $\rightarrow $. We could draw a graph to make our proof.


\begin{itemize}

\item A directed line segment has a Initial point and a Terminal point.  

\item $\overrightarrow{AB}$. An arrow upper the letters. 

\item $\vec \alpha$: bold and lower case Roman letter.  

\item Sometimes, $\vec a,\underline u$ 

\item Magnitude, size, length: $|\vec \alpha|,|\overrightarrow {AB}|$


\end{itemize}

\paragraph{Relevent Concepts} 

Two vectors are equal if and only if their magnitude and direction are the same. No matter where they start.

The vector with zero magnitude is called the \textbf{zero vector}, denoted by $\vec 0$. $\vec 0$ is the only vector is the only vector with specific direction.We have: 
$$
\overrightarrow {AB}=\vec 0 \iff A=B
$$

A vector with magnitude $1$ is called \textbf{ unit vector}.

A vector having the same length, but opposite direction of $\vec a$, is called the negative of $\vec a$, denoted by $-\vec a$. $-\vec a$. Its a whole notation, not an operation. Thus 

$$\overrightarrow  {AB}= - \overrightarrow  {BA}$$

\subsubsection{Operations}

\paragraph{Addition of vectors}

The sum of two vectors $\vec a$ and $\vec{b}$ written as

$$
\vec a +\vec b = \vec c
$$
$\vec c $ is a vector.


Defined by \textbf{triangle method}, that is
$$
\overrightarrow  {AB}+\overrightarrow  {BC}=\overrightarrow  {AC}
$$

Defined by \textbf{parallelogram  method} ...

\paragraph{Proposition of Vectors}

For vectors $\vec a$ , $\vec b$ , and $\vec c$, the addition satisfies:

\begin{itemize}

\item $\vec a+\vec b =\vec b+\vec a $ , commutative law, 

\item $(\vec a+\vec b)+\vec c= \vec a+(\vec b+\vec c) $, association law.

\item $\vec a+\vec 0=\vec a$

\item $\vec a+(-\vec a)=\vec 0$

\item Vectors with addition is an abel group. 

\end{itemize}

First two we draw a graph to prove it. Then we prove the rest ones.

3. Set $\vec a= \overrightarrow  {AB}$ and  $\vec 0 = \overrightarrow  {BB}$. Then $\vec a +\vec 0 = \overrightarrow  {AB} +\overrightarrow  {BB}=\overrightarrow {AB}= \vec a$

4. Set $\vec a=\overrightarrow {AB}$. Then $ -\vec a = \vec {BA}$. Thus, $\vec a+(-\vec a)=\vec{AB}+\vec{BA}=\vec 0 $

\paragraph{Definition} We can define the difference of two vectors $\vec a$ and $\vec b$ to be $\vec a-\vec b= \vec a +(-\vec b)$ 

\paragraph{Triangular Inequality}



For any vectors $\vec a$ and $\vec b$, we have 
$$
|\vec a +\vec b|\leq |\vec a|+|\vec b|
$$

\subsection{Scalar multiplication}

\paragraph{Definition} 
The product of vector $\vec a$ and a scalar $\lambda$, wirtten as $\lambda \vec a$, is a vector, defined by 
$$
|\lambda \vec a|= |\lambda||\vec a|
$$

\paragraph{Direction}

\begin{itemize}

\item $\lambda >0$, $\lambda \vec a$ has the same direction as $\vec a$
\item $\lambda <0$, $\lambda \vec a$ has the opposite direction as $\vec a$

\end{itemize}
\paragraph{Proposition}

\begin{itemize}
\item $\lambda \vec a =\vec 0 \iff \lambda = 0 \quad or \quad  \vec  a = \vec 0 $
\item $1 \ \vec a=\vec a,\quad (-1)\vec a = - \vec  a$
\item $\lambda(\mu \vec a ) = (\lambda \mu )\vec  a$
\item $(\lambda + \mu )\vec  a = \lambda \vec a  + \mu \vec a$ distributive law 
\item $\lambda (\vec  a + \vec  b)= \lambda \vec a + \lambda \vec  b$ distributive law 
\end{itemize}

\paragraph{Proof}

\begin{itemize}



\item $\lambda \vec a  = \vec  0  \iff |\lambda \vec a | = |\vec 0 | =0 \iff |\lambda||\vec  a|=0 \iff |\lambda |=0 \quad or \quad |\vec a|=0\iff \lambda =0 \quad \vec  a = \vec 0 $.
\item 
\item  ...
\item $|\lambda (\mu \vec a )|= |\lambda ||\mu \vec  a |\\= |\lambda||\mu||\vec  a |\\= (|\lambda ||\mu|)|\vec a|\\= |(\lambda \mu)\vec a|$

That is, $\lambda (\mu \vec  a )$ and $(\lambda \mu )\vec a$ has the same \textbf{length}.

Then we consider the \textbf{direction}.

​	Case 1. $\lambda \mu=0$, $ \vec 0 =(\lambda \mu )\vec  a= \lambda (\mu \vec a )$

​	Case 2. $\lambda \mu >0$, Trivially, $\lambda (\mu \vec  a )$has the same direction as $\vec  a $;

​												$(\lambda \mu) \vec  a$has the same direction as $\vec  a$;											

​	Case 3. $\lambda \mu <0$, Trivially, $\lambda (\mu \vec  a )$has the opposite direction of $\vec  a $;

​												$(\lambda \mu) \vec  a$has the opposite direction of $\vec  a $;			

​	Thus, we can see that $\lambda (\mu \vec  a )$ has the same direction as $(\lambda \mu) \vec  a $.

​	Hence, 
$$
(- \lambda )\vec a  =  (-1)\lambda \vec a  = (-1)(\lambda \vec a)= -\lambda \vec a 
$$

If $\lambda =0 \quad or\quad  \mu =0 \quad or \quad \vec a =\vec 0 $, it is easily to see that $(\lambda + \mu )\vec  a = \lambda \vec a  + \mu \vec a$.

consider $\lambda \neq 0 \quad or\quad  \mu \neq 0 \quad or \quad \vec a \neq \vec 0 $.

​	$1^{\circ}$  assume that $\lambda >0 , \mu >0 $ then,
$$
|(\lambda + \mu)\vec a |= |\lambda +\mu ||\vec a | 
$$
$$
= (\lambda +\mu )|\vec a | 
$$
$$
= \lambda |\vec a |+\mu |\vec a | 
$$
$$
= |\lambda||\vec a| + |\mu | |\vec  a |
$$
$$
= |\lambda \vec a | + |\mu \vec a |\quad (for \quad  \vec a \parallel \vec a) 
$$
$$
= |\lambda \vec a +\mu \vec a |
$$
​	$2^{\circ}$  If one of  $\lambda ,\mu $ is negative, we can put the terms containing the negative scalars to the other side of the equation.

​	For example: 

​	$\lambda >0 , \mu <0,\lambda +\mu <0 $ 
$$
(\lambda +\mu )\vec a = \lambda \vec a +\mu \vec a \iff -\mu \vec a  = \lambda \vec a +[-(\lambda +\mu )\vec a ]
$$
$$
\iff \lambda \vec  a +(-(\lambda + \mu ))\vec a  = (-\mu) \vec a
$$



​	(which back to the case of $\lambda >0 , \mu >0 ,\lambda + \mu >0$)



\item Case 1: $\lambda = 0$,or $\vec a = \vec 0$, or $\vec b= \vec 0 $

Case 2:$\lambda \neq 0$,or $\vec a \neq \vec 0$, or $\vec b\neq \vec 0 $

$\lambda( \vec  a + \vec  b) = \lambda \vec a +\lambda \vec b $(Graph...(similarity of triangle))


\end{itemize}

\subsection{Collinear and Coplanar Vectors}

\paragraph{Definition}

When we move some vectors to the same initial point, if they are on the same line or plane, then we say these vectors are \textbf{collinear }or \textbf{coplanar}. 

Trivially,

\begin{itemize}

\item $\vec 0 $ Is collinear with any vector. 
\item Collinear vectors must be coplanar.
\item Any two vectors must be coplanar.

\end{itemize}


\subsubsection{ Collinear Vectors}


\paragraph{Notation}

two vectors are collinear $\iff$ their directions are the same or opposite. We write $\vec a \parallel \vec b $

\paragraph{Proposition} 

For vectors $\vec a $ and $ \vec  b$ , if there exists a scalar $\lambda$ $s.t.$  $\vec b = \lambda \vec a $, then $\vec a , \vec b $ are collinear.

\paragraph{Theorem}

(Existence): Assume that $\vec a \neq \vec 0$. If $\vec a, \vec b  $ collinear, then there exists scalar $\lambda \quad s.t. \quad  \vec b = \lambda \vec a $.

\paragraph{Proof}

If $\vec b = \vec 0 $, then $\lambda =0$

Consider $\vec b \neq \vec 0 $, then 

$$
\vec{b}=|\vec{b}| \frac{\vec{b}}{|\vec{b}|}=\left\{\begin{array}{ll}
\frac{\vec{b}}{|\vec{a}|}, & \vec{a} \text { and } \vec{b} \text { have the same direction } \\
$$
$$
\frac{-|\vec{b}|}{|\vec{a}|}, & \vec{a} \text { and } \vec{b} \text { have the opposite direction }
\end{array}\right.
$$


(Uniqueness): Assume that $\vec b = \lambda ' \vec a $ Then 

$$
(\lambda -\lambda ' )\vec a  = \vec 0 \\
  For \quad \vec a \neq \vec 0 \\
  \Rightarrow  \quad \lambda = \lambda '
$$

\subsubsection{Three coplanar Vectors}

\paragraph{Proposition}
For vectors $\vec a, \vec b ,\vec c $ if there exists scalar $\lambda , \mu $, such that $\vec c = \lambda \vec a +\mu \vec b $, then $\vec a, \vec b, \vec c$ are coplanar. (Graph...)

\paragraph{Proof}

If $\vec a  = \vec 0$, then $\vec c = \lambda \vec  b$, so vectors $\vec  b \parallel \vec  c$, then $\vec  a, \vec  b, \vec  c $ are Coplanar;

If $\vec  a \neq \vec  0 $, then consider two cases:

Case 1: $\vec  a \parallel \vec  b$. Thus, there exist $k$ s.t. $\vec  b = k \vec  a $. 

Hence, $\vec  c = \lambda \vec  a + \mu \vec  b = \lambda \vec  a +\mu k \vec  a  = (\lambda + \mu k ) \vec  a $ .

Therefore, $\vec c \parallel \vec  a$. 

Case 2: $\vec  a \not \parallel \vec  b$, then $\vec  c$ is the diagrod of the parallelogram formed by $\lambda \vec  a, \mu \vec  b$

\paragraph{Theorem}

Assume that $\vec a, \vec b $ are \textbf{not collinear}, then for any vector $\vec c $  on the plane determined by $\vec a$ and $\vec b $ , there exist \textbf{unique} scalars $\lambda , \mu \quad s.t. \quad $


$$
\vec c = \lambda \vec a + \mu \vec b
$$

\paragraph{Proof}

$1^{\circ}$We first prove the existence of $\lambda, \mu $. 

(Graph...)

We can wrote $\vec c = \vec  c_1+ \vec  c_2, \vec c_1 \parallel \vec a, \vec c_2 \parallel \vec  b$, 

Since $\vec a,\vec  b \neq \vec  0$, 

$\exists \lambda,\mu,\quad \vec c_1 = \lambda \vec a, \vec c_2 = \mu \vec b$

Then, $\vec c = \lambda \vec  a+\mu \vec  b$

$2^{\circ}$Suppose that $\vec c = \lambda ' \vec  a+\mu '\vec  b $

We see that $(\lambda - \lambda ')\vec  a +(\mu - \mu ')\vec b  = \vec  0$

If $\lambda \neq \lambda '$, then 
$$
\vec a  = - \frac {\mu - \mu '}{\lambda - \lambda  ' }\vec b
$$
then $\vec a \parallel \vec b$, which is a contradiction.

thus, $\lambda = \lambda '$, we have $(\mu - \mu ')\vec b = \vec 0$

Since $\vec b \neq \vec 0 , \mu = \mu '$


\subsubsection{Three Non-coplanar Vectors}




\paragraph{Theorem} If $\vec a, \vec b , \vec c$ are not coplaner, then or any vector $\vec u$, there exist unique scalars $\lambda,\mu,\nu$,s.t.
$$
\vec u =\lambda \vec a+ \mu \vec b + \nu \vec c
$$
\paragraph{Proof} 
$1^{\circ}$Existence of $\lambda,\mu ,\nu $

​			(Graph...)

​			$2^{\circ}$Assume that 
$$
\vec u = (\lambda  - \lambda^*)\vec a + (\mu - \mu^*)\vec b + (\nu - \nu^*)\vec c
$$
​				Suppose $\lambda \neq \lambda^*$ , then 
$$
\vec a  = - \frac{\mu - \mu^*}{\lambda- \lambda^*}\vec b - \frac {\nu - \nu^*}{\lambda - \lambda^*}\vec c
$$
​				Showing that $\vec a,\vec b \vec c $. are coplaner. This is a contradiction. 

​				Hence, $\lambda = \lambda ^*$,We have 
$$
(\mu - \mu ^*)\vec b + (\nu - \nu ^*)\vec c = \vec 0 
$$
​				Using the similar argument in the proof of the last theorem, we know that  $\mu = \mu ^*,\nu = \nu ^*\quad (\vec b \not \parallel \vec c)$



\subsubsection{Three Points on the same line}

Point $C$ is on the line segment  $AB$ if and only if there exist scalar $\lambda ,\mu \geq 0 \quad (\lambda + \mu = 1)$ s.t. 
$$
\overrightarrow{OC} = \lambda \overrightarrow{OA} + \mu \overrightarrow{OB}
$$
for any point $O$.


\paragraph{Proof}

Point $C$ is on $AB \iff \overrightarrow{AC} \parallel \overrightarrow{AB}$.

$$
\iff |\overrightarrow{AC}|\leq |\overrightarrow{AB}| 
$$

$$
\iff \exists \mu \in [0,1]  \quad s.t. \quad  \overrightarrow{AC}= \mu \overrightarrow{AB} 
$$

$$
\iff \overrightarrow {OC}-\overrightarrow {OA} = \mu (\overrightarrow{OB}- \overrightarrow{OA}) 
$$

$$
\iff \overrightarrow{OC} = (1-\mu)\overrightarrow{OA}+\mu \overrightarrow{OB}
$$

$$
\iff \overrightarrow{OC} = \lambda \overrightarrow{OA}+ \mu \overrightarrow{OB},\quad (\lambda + \mu  = 1)
$$




\subsection{Affine Coordinate System}

\subsubsection{ Coordinate System }

By a theorem of \textbf{1-2}, if $\vec e_1,\vec e_2,\vec e_3$ are not coplanar, then for any vector $\vec a$, there exist unique scalar $a_1,a_2,a_3$ s.t.
$$
\vec a  = a_1 \vec e_1+ a_2 \vec e_2 + a_3\vec e_3
$$
The ordered tripe$(a_1,a_2,a_3)$ is called the coordinate of $\vec a $.

It can be show that the mapping $\vec a \mapsto   (a_1,a_2,a_3)$ is a one-to-one  correspondence.

We simply write $\vec a  = (a_1,a_2,a_3)$ 

Obviously, $\vec e_1 = (1,0,0),\quad \vec e_2 = (0,1,0), \quad \vec e_3 = (0,0,1)$

Origin + Basis = Coordinate System
$$
[O,\vec e_1,\vec e_2,\vec e_3]
$$
If $\vec e_1 \bot \vec e_2, \vec e_2 \bot \vec e_3, \vec e_1 \bot \vec e_3$ and $\vec e_1,\vec e_2, \vec e_3$ are unit vectors, 
then $[O,\vec e_1,\vec e_2,\vec e_3]$ is called the Cartesian coordinate system.

\subsubsection{Algebraic Operations Using Coordinate}

\paragraph{ Theorem}  In an affine coordinate system$[O,\vec e_1,\vec e_2,\vec e_3]$, 
assume that $\vec a = (a_1, a_2, a_3)$ and $\vec b = (b_1, b_2, b_3)$, then,

$$\vec a +\vec b = (a_1+b_1,a_2+b_2,a_3+b_3)$$

\paragraph{ Proof}
$$
\vec a + \vec b  
$$
$$
\\ = (a_1\vec e_1 + a_2 \vec e_2 + a_3\vec e_3 )+ (b_1\vec e_1 + b_e \vec b_2 + b_3\vec e_3 ) 
$$
$$
\\ = (a_1+b_1)\vec e_1 + (a_2+b_2) \vec e_2 + (a_3+b_3) \vec e_3
$$

$k \vec a  = (ka_1,ka_2,ka_3)$

\paragraph{Proof}
$$
k \vec a 
\\ = k(a_1\vec e_1 + a_2 \vec e_2 + a_3\vec e_3)
\\ = k a_1\vec e_1 + k a_2 \vec e_2 + k a_3\vec e_3
$$


Corollary Coordinary of $\overrightarrow {AB}$ = Coordinate of $B$  - Coordinate $A$
$$
\overrightarrow{AB} = \overrightarrow{OB} - \overrightarrow {OA}
$$

now let's assume that we have three vectors $\vec a ,\vec b , \vec c$,


\subsubsection{Scalar Products of Vectors}


\paragraph{ Definition}
The scalar product (or inner product, or dot product) of two vectors $\vec a$ and $\vec b$ is a scalar, denoted by $\vec a \cdot \vec b$, and is defined by 
$$
\vec a \cdot \vec b  = |\vec a||\vec b|\cos<\vec a,\vec b >
$$
where $ <\vec a,\vec b>\in [0,\pi] $

\paragraph{ Proposition}

\begin{itemize}
  \item $\vec a \cdot \vec a  = |\vec a |^2 \geq 0 \Rightarrow |\vec a| = \sqrt{\vec a \cdot \vec a}$
  \item $\vec a \cdot \vec b  = \vec b \cdot \vec a$ (commutative)
  \item $\vec a \cdot \vec b = 0 \iff \vec a \bot \vec b \quad (\vec 0$ is perpendicular to any vectors)
  \item Cauchy-Schwarz inequality $|\vec a \cdot \vec b| \leq |\vec a|| \vec b|$, the equation holds if and only if $\vec a$ is collinear with $\vec b$.
\end{itemize}

\paragraph{Theorem}
\begin{itemize}
  \item $(\lambda \vec a) \cdot \vec b = \lambda (\vec a \cdot \vec b) = \vec a\cdot (\lambda \vec b)$
  \item $\vec a \cdot (\vec b + \vec c) = \vec a \cdot \vec b + \vec a \cdot \vec c$
\end{itemize}

\paragraph{Example} Law of Cosine

In triangle $ABC$
$$
\vec c \cdot \vec c = (\vec a + \vec b)^2 = |\vec a|^2+|\vec b|^2+\vec a \cdot \vec b = |\vec a|^2 + |\vec b|^2 - 2|\vec a||\vec b| \cos {C}
$$

\paragraph{Example} Three height of $\triangle ABC$ concurrent. Claim that $\overrightarrow{CO}\bot \overrightarrow{AB}$.
$$
\overrightarrow {CO} \cdot \overrightarrow{AB} = ... = 0
$$

\paragraph{Calculate $\vec a \cdot \vec b$ Using Coorinates}

In an affine coordinate system $[O,\vec e_1,\vec e_2,\vec e_3]$. Assume that $\vec a = (a_1,a_2,a_3), \vec b = (b_1,b_2,b_3)$, Then 
$$
\vec a \cdot \vec b = \sum a_i \vec e_i \cdot \sum b_i \vec e_i = ... 
$$
In particular, in the Cartesian coordinate system, 
$$
\vec a \cdot \vec b = \sum a_ib_i
$$
Moreover, 
$$
|\vec a| = \sqrt{\sum a_i^2}
$$
$$
\cos <\vec a, \vec b> = \frac{\vec a\cdot \vec b}{|\vec a||\vec b|}= \frac{\sum a_ib_i}{\sqrt{\sum a_i^2} \sqrt{\sum b_i^2}}
$$
\paragraph{Example} In a regular tentrahedron $ABCD$, $E$ is the midpoint of $AB$ and $F$ is the midpoint of $CD$.
Every length equals to $a$. Find $|\overrightarrow{EF}|$.

Consrruct an affine c.s. $[A,\overrightarrow{AB},\overrightarrow{AC},\overrightarrow{AD}]$.
Write 
$$
\vec e_1 = \overrightarrow{AB}, \vec e_2 = \overrightarrow{AC},\vec e_3= \overrightarrow{AD}
$$
$$
\overrightarrow{AF} = \frac{1}{2}(\vec e_2+ \vec e_3)= (0,\frac{1}{2},\frac{1}{2})
$$
$$
\overrightarrow{AE}= ...
$$
Since $\vec e_i^2= a^2, \vec e_i \cdot \vec e_j = \frac{1}{2} a^2$
$$
\overrightarrow{EF}^2= (-\frac{1}{2},\frac{1}{2},\frac{1}{2})^2 = \frac{a^2}{2}
$$
Thus $|\overrightarrow{EF}|= \frac{\sqrt{2}}{2}a$

\subsubsection{Vector Product of Vectors}

\paragraph{Definition}
The vector product (or cross product) of two vectors$\vec a$ and $ \vec b$ is a vector, denoted by $\vec a \times \vec b$,
and is defined by 
$$
\text{magnitude:} |\vec a \times \vec b | = |\vec a| |\vec b| \sin <\vec a,\vec b>  \quad
\text{direction: Right hand rule}
$$
Note: $|\vec a \times \vec b| = $ area of parallelogram formed by $\vec a, \vec b$ 

\paragraph{Proposition}

\begin{itemize}
  \item $\vec a \times \vec b = - \vec b \times \vec a $
  \item $\vec a \times \vec b = \vec 0 \iff \vec a \parallel \vec b$
  \item $(\lambda \vec a)\times \vec b  = \lambda (\vec a \times \vec b) = \vec a \times (\lambda \vec b)$
  \item $\vec a \times (\vec b + \vec c) = \vec a \times \vec b+ \vec a \times \vec c$
\end{itemize}

\paragraph{Introdection to Determinant}

$$
\left|\begin{array}{ll}
  a_{1,1} & a_{1,2} \\
  a_{2,1} & a_{2,2}
  \end{array}\right|=a_{1,1} a_{2,2}-a_{1,2} a_{2,1}
$$

$$
\begin{array}{|lll|l}
  a_{1,1} & a_{1,2} & a_{1,3} \\
  a_{2,1} & a_{2,2} & a_{2,3} \\
  a_{3,1} & a_{3,2} & a_{3,3}
  \end{array} 
  =a_{1,1} a_{2,2} a_{3,3}+a_{1,2} a_{2,3} a_{3,1}+a_{1,3} a_{2,1} a_{3,2}-a_{1,3} a_{2,2} a_{3,1}-a_{1,1} a_{2,3} a_{3,2}-a_{1,2} a_{2,1} a_{3,3}\\
  =...
$$
...(missed a class)\dots

\section{Planes and straight Lines}
\subsection{Equation and its Graphs}
\paragraph{Example}
In the Cartesian c.s.
\begin{itemize} 
  \item Sphere centered at $(a,b,c)$ with radius $r$: $(x-a)^2+(y-b)^2+(z-c)^2=r^2$.
  \item Unit circle centered at the origin on the $xy$ plane: $x^2+y^2=1,z=0$.
\end{itemize}
Graph - Point/coordinate - equation
In general,
\begin{itemize}
  \item Surface $F(x,y,z)=0$.
  \item Curve 
  $\left\{\begin{array}{l}
    F(x, y, z)=0 \\
    G(x, y, z)=0
    \end{array}\right.$
  \item Parametric equation 
    $\left\{\begin{array}{l}
    x=f(t) \\
    y=g(t) \\
    z=h(t)
    \end{array}\right.
  $\\
  circle on the $xy$ plane center at the origin
  $\left\{\begin{array}{l}
    x=a\cos \theta \\
    y=a \sin \theta\\
    z=0
    \end{array}\right.
  $\\
  Helix
  $\left\{\begin{array}{l}
    x=a\cos \theta \\
    y=a \sin \theta\\
    z=\theta
    \end{array}\right.
  $\\
\end{itemize}
\subsection{Planes in an Affine Coordinate System}
\subsubsection{Equation of a Plane}
In an affine coordinate system, assume that plane $\pi$ passes
through point $M_0(x_0,y_0,z_0)$, and is paralleled to two vectors 
$\vec u_1=(X_1,Y_1,Z_1)$ and $\vec u_2=(X_2,Y_2,Z_2)$. Point $M(x,y,z)$
is on the plane $\pi$.
$$
\iff \overrightarrow{M_0M},\vec a,\vec b \quad \text{are coplanar}
$$
$$
\iff
\left|\begin{array}{ccc}
  x-x_{0} & y-y_{0} & z-z_{0} \\
  X_{1} & Y_{1} & Z_{1} \\
  X_{2} & Y_{2} & Z_{2}
  \end{array}\right|=0
$$
$$
\Rightarrow Ax+By+Cz+D=0
$$
where
$$
\begin{aligned}
  A=\left|\begin{array}{ll}
  Y_{1} & Z_{1} \\
  Y_{2} & Z_{2}
  \end{array}\right|, & B=\left|\begin{array}{ll}
  Z_{1} & X_{1} \\
  Z_{2} & X_{2}
  \end{array}\right|, \quad C=\left|\begin{array}{ll}
  X_{1} & Y_{1} \\
  X_{2} & Y_{2}
  \end{array}\right| \\
  D &=-\left|\begin{array}{lll}
  x_{0} & y_{0} & z_{0} \\
  X_{1} & Y_{1} & Z_{1} \\
  X_{2} & Y_{2} & Z_{2}
  \end{array}\right|
  \end{aligned}
$$
\paragraph{Conclusion}Now We have shown that every plane in an affine coordinate system 
has the equation in the form of
$$
Ax+By+Cz+D=0.
$$
Since $\vec a \times \vec b \neq \vec 0$,
$A,B,C$ are not all $0$s.


On the other hand, consider $kx+my+nz+p=0$, 
$$
\Rightarrow 
\left|\begin{array}{ccc}
  x+\frac{p}{k} & y & z \\
  -m & k & 0 \\
  -\frac{n}{k} & 0 & 1
  \end{array}\right|=0
$$

showing that the point on this surface that passes through point $(-\frac{p}{k},0,0)$
and is parallel to vectors $(-m,k,0)$ and $(-n,0,k)$. This
surface is a plane.

\paragraph{Example} In an affine c.s., plane $\pi$ passes 
$M_1(a,0,0),M_2(0,b,0),M_3(0,0,c)(a,b,c\neq 0)$,
\subparagraph{Solution}
$$
\begin{array}{lll}
  \left|\begin{array}{ccc}
  x-a & y & z \\
  -a & b & 0 \\
  -a & 0 & c
  \end{array}\right|=0 \\
  \end{array}
$$
$$
\Rightarrow \frac{x}{a}+\frac{y}{b}+\frac{z}{c}=1
$$

Alternatively, assume the euqation of $\pi$ is $Ax+By+Cz+D=0$,
$$
A=-\frac{D}{a}, \quad B=-\frac{D}{b}, \quad C=-\frac{D}{c}
$$
$$
 \Rightarrow \frac{Dx}{a}+\frac{Dy}{b}+\frac{Dz}{c}=D
$$
If $D=0$, $\pi$ passes through the origin.
Thus $D\neq 0$
$$
 \Rightarrow \frac{x}{a}+\frac{y}{b}+\frac{z}{c}=1
$$

\paragraph{Theorem} In an afifne coordinate system, vector $(a,b,c)$
is parallel to plane \\$\pi : Ax+By+Cz+D=0$ if and only if
$$
Aa+Bb+Cc=0
$$
\paragraph{Proof} 
Pick a point $M_{0}\left(x_{0}, y_{0}, z_{0}\right)$ on
the plane $\pi$, 
$$
A x_{0}+B y_{0}+C z_{0}+D=0
$$
Pick 
$M\left(x_{0}+k, y_{0}+m, z_{0}+n\right)$ 
and a vector $\vec r=\overrightarrow{M_{0} M},$ 
then $r\parallel \pi \iff M \in \pi,$ which is 
$$
A\left(x_{0}+k\right)+B\left(y_{0}+m\right)+C\left(z_{0}+n\right)+D=0
$$
Thus $A a+B b+C c=0 .$

\paragraph{Consequently}
\begin{itemize}
  \item $\pi \parallel x \quad axis \iff A=0$
  \item $\pi \parallel y \quad axis \iff B=0$
  \item $\pi \parallel z \quad axis \iff C=0$
  \item $\pi \parallel xy \quad plane \iff A=0,B=0$
\end{itemize}

\paragraph{Theorem} In an affine c.s., plane 
$$
\begin{array}{l}
  \pi_{1}: A_{1} x+B_{1} y+C_{1} z+D_{1}=0 \\
  \pi_{2}: A_{2} x+B_{2} y+C_{2} z+D_{2}=0
  \end{array}
$$
\begin{itemize}
  \item 
    $\pi_{1} \parallel \pi_{2} \Longleftrightarrow A_{1}: A_{2}=B_{1}: B_{2}=C_{1}: C_{2} ;$
  \item $\pi_{1}=\pi_{2} \Longleftrightarrow A_{1}: A_{2}=B_{1}: B_{2}=C_{1}: C_{2}=D_{1}: D_{2}$
  \item $\pi_{1} $intersect $\pi_{2}  \iff (A_1,B_1,C_1)$ is not propotional to $A_2,B_2,C_2$.
\end{itemize}

\paragraph{Proof}
Set 
$$
\vec u_1 = (-B_1,A_1,0)
$$
$$
\vec v_1 = (-C_1,0,A_1)
$$
which are all parallel to $\pi_1$,

Without loss of genrealarity,  assume $A_1\neq 0\quad (A_2\neq 0)$

If $(A_1,B_1,C_1)=k(A_2,B_2,C_2)$, Then 
$$
(-B_1)A_2+A_1B_2+0\cdot C_2=0 \Rightarrow \vec u_1 \parallel \pi_2
$$
$$
(-C_1)A_2+0\cdot B_2+ A_1C_2=0 \Rightarrow \vec v_1 \parallel \pi_2
$$
$$
\vec u_1 \not \parallel \vec v_1 \Rightarrow \pi_1 \parallel \pi_2
$$
\subsubsection{Relation of three Planes}
\paragraph{Theorem} In an affine c.s.
let
$$
\begin{array}{l}
\pi_{1}: A_{1} x+B_{1} y+C_{1} z+D_{1}=0 \\
\pi_{2}: A_{2} x+B_{2} y+C_{2} z+D_{2}=0 \\
\pi_{3}: A_{3} x+B_{3} y+C_{3} z+D_{3}=0
\end{array}
$$
$\pi_{1}, \pi_{2}, \pi_{3}$ are concurrent at one point if and only if 
$$
\left|\begin{array}{lll}
A_{1} & B_{1} & C_{1} \\
A_{2} & B_{2} & C_{2} \\
A_{2} & B_{3} & C_{3}
\end{array}\right| \neq 0
$$
\paragraph{Proof} $\pi_{1}, \pi_{2}, \pi_{3}$ concurrent at one point 
if and only if
$$
\left\{\begin{array}{l}
A_{1} x+B_{1} y+C_{1} z+D_{1}=0 \\
A_{2} x+B_{2} y+C_{2} z+D_{2}=0 \\
A_{3} x+B_{3} y+C_{3} z+D_{3}=0
\end{array} \right.
$$

\section{Straight lines in an Affine Coordinate System}
\subsection{Equation of lines}
Assume that line $l$ passes through point $M_0(x_0,y_0,z_0)$ and parallel to the vector $\vec u = (a,b,c)$. Point $M(X,Y,Z)$is on line $l$,
$$
\iff \overrightarrow{MM_0} \parallel \vec u 
$$
$$
\iff \overrightarrow{MM_0} = \lambda \vec u 
$$

So we got the Parametric equation of the line (A point and a direction)
$$
\left\{\begin{array}{l}
x=x_{0}+\lambda X \\
y=y_{0}+\lambda Y \\
z=z_{0}+\lambda Z
\end{array}\right.
$$

if $a,b,c \neq 0$

$$
\frac{x-x_{0}}{X}=\frac{y-y_{0}}{Y}=\frac{z-z_{0}}{Z}
$$

Aline can also be determined by two non-parallel planes.
...
q

$$
\left\{\begin{array}{l}
A\ Plane\\
Another\ Plane 
\end{array}\right.
$$
\subsection{Relationship between a Plane and a Line}
$$
l :\frac{x-x_{0}}{X}=\frac{y-y_{0}}{Y}=\frac{z-z_{0}}{Z}
$$
$$
\pi : Ax+By+Cz+D=0
$$
...


\section{Relationship between two lines}
$$
\begin{aligned}
&l_{1} \parallel l_{2} \Longleftrightarrow u_{1} \parallel u_{2}
$$
$$
&l_{1}, l_{2} \quad \text { coplanar } \Longleftrightarrow\left(\overrightarrow{M_{1} M_{2}}, u_{1}, u_{2}\right)=0,
$$
$$
&l_{1}, l_{2} { are\ the\ same\ line } \Longleftrightarrow \overrightarrow{M_{1} M_{2}}, u_{1}, u_{2} {are \ collinear. }
$$
\end{aligned}
kk//
\subsection{Sheef of Planes}
Assume that line $l$ is given by 
$$
\left\{\begin{array}{l}
A_{1} x+B_{1} y+C_{1} z+D_{1}=0 \\
A_{2} x+B_{2} y+C_{2} z+D_{2}=0
\end{array}\right.
$$
 Since $\pi_1 \neq \pi_2$, we know $(A_1,B_1,C_1)$ is not proportional to $(A_2,B_2,C_2)$. Thus, for any $\lambda, \mu$, there are not both $0$s
 
 $$
 \lambda A_{1}+\mu A_{2}, \quad \lambda B_{1}+\mu B_{2}, \quad \lambda C_{1}+\mu C_{2}
 $$
 Hence, the equation 
 $$
 \lambda\left(A_{1} x+B_{1} y+C_{1} z+D_{1}\right)+\mu\left(A_{2} x+B_{2} y+C_{2} z+D_{2}\right)=0
 $$ ...
 which is a plane.
 
\paragraph{Definiton} \{$S=\textit{planes in the form of }  \lambda\left(A_{1} x+B_{1} y+C_{1} z+D_{1}\right)+\mu\left(A_{2} x+B_{2} y+C_{2} z+D_{2}\right)=0, \lambda, \mu \textit{are not both 0s.}\}$
 
 \paragraph{Definition}$T = \{\textit{ planes that passes through }l\}$. We will prove $S=T$
 \paragraph{Proof}
 1. Claim $S$ ...
 
 \subsection{An Example}..
 
 \section{Planes and Lines in the Cartesian Coordinate system}
 







\end{document}
